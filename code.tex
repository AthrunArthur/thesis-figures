
\newsavebox{\thenlisting}
\begin{lrbox}{\thenlisting}% Store first listing
\begin{lstlisting}
para<int> a;
para<char> b;
para<float> c;
para<double> d;
(a || (b && c) || d).then(func);
\end{lstlisting}
\end{lrbox}


\newsavebox{\samplelisting}
\begin{lrbox}{\samplelisting}
\begin{lstlisting}
ff::paracontainer pc;
while(...){
  int i = next();
  ff::para<doble> a;
  ff::para<void> b;
  a([i](){return fa(i);});
  b[a]([i](double d){fb(i, d);});
  pc.add(b);
}
ff_wait(all(pc));
\end{lstlisting}
\end{lrbox}

\newsavebox{\osamplelisting}
\begin{lrbox}{\osamplelisting}
\begin{lstlisting}
ff::paracontainer pc;
while(...){
  int i = next();
  (*{\color{red}{\tt optimized\_para\_a}}*)(pc, i);
}
(*{\color{red}{\tt clear\_para\_a}}*)(pc);
ff_wait(all(pc));
\end{lstlisting}
\end{lrbox}

\newsavebox{\optfunclisting}
\begin{lrbox}{\optfunclisting}
\begin{lstlisting}
  void optimized_para_a(ff::paracontainer & pc, int & ii){
    buf.push_back(ii); //global std::vector<int> buf;
    if(buf.size() < thres_hold)
      return ;
    ff::para<std::vector<double> > a;
    a([buf](){
      std::vector<double> res;
      for(auto i : buf){
        double d = fa(i);
        res.push_back(d);
      }
    });
    ff::para<> b;
    b[a]([buf](std::vector<double> ds){
      for(size_t i = 0; i<buf.size(); ++i)
        fb(ds[i], buf[i]);
      });
    pc.add(b);
    buf.clear();
  }
\end{lstlisting}
\end{lrbox}

\newsavebox{\depexample}
\begin{lrbox}{\depexample}
\begin{lstlisting}[mathescape]
while(...){
  ff::para<void> a,b,c,d,e;
  a(...); b(...);
  c[a && b](...);
  d[c](...); e[a](...);
  ff_wait(a&&b&&c&&d&&e);
}
\end{lstlisting}
\end{lrbox}

\newsavebox{\waitandthen}
\begin{lrbox}{\waitandthen}
  \begin{lstlisting}[mathescape]
template <class E1, class E2> class wait_and{
 typedef typename E1:ret_type T1;
 typedef typename E2:ret_type T2;
 typedef std::tuple<T1, T2> ret_type;
 E1 & m1;
 E2 & m2;
 ...

 ret_type get(){return ret_type(m1.get(),m2.get());}

 template<class F>
 auto call_with_ret_vals(F && f) ->typename func_ret_type<f>::type{ return f(m1.get(), m2.get());}

 template <class F>
 auto then(F && f) ->typename
  std::enable_if<is_callable_with_args<f, T1_t, T2_t>::value, func_ret_type<f>::type>::type {...}
};
  \end{lstlisting}
\end{lrbox}


% ======= STORE/BOX LISTINGS =======
\newsavebox{\tasklisting}
\begin{lrbox}{\tasklisting}% Store first listing
\begin{lstlisting}
 int g_counter = 0;
 void task_fun(int i){
    g_counter ++;
    ms[g_counter%2].lock();
    if( is_odd(i)){
        ms1[i].lock();
    }else{
        ms2[i].lock();
    }
    ...
 }
\end{lstlisting}
\end{lrbox}

\newsavebox{\tasklocklisting}
\begin{lrbox}{\tasklocklisting}
\begin{lstlisting}
  vido task_level_lock::lock(){
    if(m_mutex.try_lock()) return ;
    ucontext_t * ctx = ctx_pool.alloc();
    getcontext(ctx);
    if(m_mutex.try_lock()){ ctx_pool.free(ctx)}
    else{
      ctx_task t(ctx);
      enqueue(t);
      setcontext(thread_entry_point);
    }
  }
\end{lstlisting}
\end{lrbox}

\newsavebox{\checkerlisting}
\begin{lrbox}{\checkerlisting}% Store second listing
\begin{lstlisting}
 bool task_fun_checker(int i){
    bool res = false;
    if(is_odd(i)){
        res = res || ms1[i].is_locked();
    }else{
        res = res || ms2[i].is_locked();
    }
    ....
    return res;
 }
\end{lstlisting}
\end{lrbox}

\newsavebox{\pushfrontlisting}
\begin{lrbox}{\pushfrontlisting}
\begin{lstlisting}
void push_front(task * t){
  m_status.fetch_or(1<<16);
  while(m_status.load() != 1<<16)
    std::this_thread::yield();

  //!push t here

  m_status.store(0);
}
\end{lstlisting}
\end{lrbox}

\newsavebox{\steallisting}
\begin{lrbox}{\steallisting}
\begin{lstlisting}
  task * steal(){
  uint16_t expect = 0;
  while(!m_status.compare_exchange_strong(
   expect, expect + 1)){
    if(expect & 1<<16) return NULL;
      expect = expect & ((1<<16) - 1));
    }//!original steal here
  m_status.fetch_sub(1);
  //!return here.
}
\end{lstlisting}
\end{lrbox}

