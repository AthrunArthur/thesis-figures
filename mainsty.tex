\usepackage{xeCJK}

%\ifStandaloneMode
%\else
%\usepackage{makecell}
%\fi

\usepackage{listings}
\usepackage{afterpage}
\usepackage{multirow}
\usepackage{CJKfntef}
\usepackage[perpage,symbol*]{footmisc}
\usepackage{tikz}
\usepackage{tikzscale}

\usepackage{pgfplots}
\usepackage{fancyvrb}
\usepackage{tabularx}
\usepackage{color, colortbl}
\usepackage[section]{placeins}
\usepackage{alltt}
\usepackage[linesnumbered, longend, ruled,vlined]{algorithm2e}
\usepackage{algorithmic}
\usepackage{subfigure}
%\usepackage[ruled, linesnumbered]{algorithm2e}
\renewcommand{\algorithmcfname}{算法}

%\usepackage{amsthm}







% addpackages, by jpwu
\usepackage[squaren]{SIunits}
\usepackage{float}
%带圈的数字
\usepackage{pifont}
%解决中文连字符左边有空格的问题
\normalspacedchars{-}


\newenvironment{CenteredBox}{%
\begin{Sbox}}{% Save the content in a box
\end{Sbox}\centerline{\parbox{\wd\@Sbox}{\TheSbox}}}% And output it centered

\newcolumntype{C}{>{\centering\arraybackslash}X}
\newcolumntype{L}[1]{>{\raggedright\let\newline\\\arraybackslash\hspace{0pt}}m{#1}@{\,}}
\newcolumntype{T}[1]{>{\centering\let\newline\\\arraybackslash\hspace{0pt}}m{#1}@{}}
\newcolumntype{R}[1]{>{\raggedleft\let\newline\\\arraybackslash\hspace{0pt}}m{#1}}

\newenvironment{mcode}{\begin{alltt}}{\end{alltt}}

\ifdefined\StandaloneMode
%nothing here
\else
\lstset{emph={long, class, return, public, const, if, else, using, namespace, struct, for, int,char, typedef, void, double, float, static_cast, new},emphstyle=\color{blue}\sffamily, 					basicstyle=\ttfamily,
  moredelim=[is][\color{red}]{^}{^}}

\lstset{language=C++,
                basicstyle=\ttfamily\footnotesize,
                keywordstyle=\color{blue}\ttfamily\footnotesize,
                stringstyle=\color{red}\ttfamily\footnotesize,
                commentstyle=\color{green}\ttfamily\footnotesize,
                morecomment=[l][\color{magenta}]{\#}
}
\lstset{%
  escapeinside={(*}{*)},%
}
\fi

%\newenvironment{CenteredBox}{%
%\begin{Sbox}}{% Save the content in a box
%\end{Sbox}\centerline{\parbox{\wd\@Sbox}{\TheSbox}}}% And output it centered


\newcommand{\bh}[1]{\texttt{#1}}
\newcommand{\cpp}[1]{\texttt{#1}}
\newcommand{\code}[1]{\texttt{#1}}
\newcommand{\cat}[1]{{\it #1}}
\newcommand{\state}[1]{\textsf{#1}}
\newcommand{\figfont}[1]{{\footnotesize #1}}
\newcommand{\disscus}[1]{{\color{red}#1}}
\newcommand{\msg}[1]{{\color{red}{\bf #1}}}

\newcommand{\todo}[1]{{\color{blue}#1}}
\newtheorem{finding}{发现}

\newcommand{\tbbpatch}{\textsc{P-TBB}}
\newcommand{\name}{\textsc{FF}}
\newcommand{\fname}{Function Flow}
\newcommand{\sslink}{SSLink}
\newcommand{\msusfont}[1]{{\textsl #1}}
\newcommand{\msus}{\textsl{plbcr}}
\newcommand{\exfigscale}{0.92}
\newcommand{\cmfigscale}{0.85}


\newcommand{\cmark}{\ding{51}}%
\newcommand{\false}{\code{false}}
\newcommand{\true}{\code{true}}
\newcommand{\xmark}{\ding{55}}%
\newcommand{\tmark}{\ding{66}}


\definecolor{kwcolor}{rgb}{.5,.3,.0}
\definecolor{fpcolor}{rgb}{.0,.3,.5}
\newcommand{\kw}[1]{{\color{kwcolor}{#1}}}
\newcommand{\fkw}[1]{{\kw{\figfont{#1}}}}
\newcommand{\dexpr}{{\textmd{D-Expr}}}
\newcommand{\type}[1]{\code{#1}}
\newcommand{\method}[1]{\code{#1}}

\newcommand{\fp}[1]{{\color{fpcolor} {#1}}}
\newcommand{\fun}[1]{{\color{red} {#1}}}
\newcommand{\task}[1]{\texttt{$#1$}}
\newcommand{\fcode}[1]{\figfont{\texttt{#1}}}
\newcommand{\bench}[1]{{\footnotesize{\textsf{#1}}}}

\newsavebox{\codebox}






\newcommand{\pushfrontalgo}{%
  \begin{algorithm}[H]
    \SetAlgoLined
\SetKwData{Left}{left}\SetKwData{This}{this}\SetKwData{Up}{up}
\SetKwFunction{Union}{Union}\SetKwFunction{FindCompress}{FindCompress}
%\SetKwInOut{Input}{Input}\SetKwInOut{Output}{Output}
\SetKw{KwBreak}{Break}
\KwIn{The array in the queue, $Q$;}
\KwIn{The queue's head, $head$;}
\KwIn{The queue's tail, $tail$;}
\KwIn{The queue's length, $len$;}
\KwIn{The task to push into the queue, $t$;}
\KwIn{Atomic boolean variable to mark if there is any thief, $is\_stealing$;}
\KwIn{Atomic boolean variable to mark this process is ongoing,
  $is\_pushing\_front$;}
\KwOut{Push the task $t$ into the queue}

$is\_pushing\_front$ $\leftarrow$ true \;
\While{$is\_stealing$} ;
\eIf{$head$ == 0}{
  $head \leftarrow$ $ len$ \;
  $tail \leftarrow$ $ tail + len $ \;
}{}
$Q[(head - 1)] \leftarrow$ $t$ \;
$head \leftarrow$ $head-1$ \;
\caption{The \cpp{push\_front} algorithm of for the work-stealing queue.}
\label{alg:pushfront}
\end{algorithm}}

\newcommand{\takeorstealalgo}{%
\begin{algorithm}[H]
  \SetAlgoLined
\SetKwData{Left}{left}\SetKwData{This}{this}\SetKwData{Up}{up}
\SetKwFunction{Union}{Union}\SetKwFunction{FindCompress}{FindCompress}
\SetKwInOut{Input}{输入}\SetKwInOut{Output}{输出}
%\SetKwInOut{Input}{Input}\SetKwInOut{Output}{Output}
\SetKw{KwBreak}{Break}
\Input{当前线程任务队列\code{WSQ};}
\Input{当前线程挂起任务队列\code{STQ};}
\Input{当前线程计数器\code{this};}
task = WSQ.pop()\;
\eIf{is\_valid(task)}{
  \KwRet task\;
}
{
  c ++\;
  \eIf {c\&1 == 0}{
    task = steal\_task()\;
    \eIf{is\_valid(task)}{
      \KwRet task\;
    }
    {
      \KwRet STQ.pop()\;
    }
  }
  {
    task = STQ.pop()\;
    \eIf{is\_valid(task)}{
      \KwRet task\;
    }
    {
      \KwRet steal\_task()\;
    }
  }
}
\end{algorithm}}

\newcommand{\thenalgo}{%
\begin{algorithm}[H]
  \SetAlgoLined
\SetKwData{Left}{left}\SetKwData{This}{this}\SetKwData{Up}{up}
\SetKwFunction{Union}{Union}\SetKwFunction{FindCompress}{FindCompress}
\SetKwInOut{Input}{输入}\SetKwInOut{Output}{输出}
%\SetKwInOut{Input}{Input}\SetKwInOut{Output}{Output}
\SetKw{KwBreak}{Break}
\Input{依赖表达式回调函数\code{func};}
\Output{依赖表达式的指针\code{this}。}
%\Begin{
\While{!this->{check\_if\_over}()}{
  task = runtime::take\_or\_steal\_one\_task()\;

  \eIf {is\_valid(task)}{
    runtime::execute\_task(task)\;
  }{
    runtime::yield()\;
  }
}
call\_with\_ret\_vals(func)\;
\label{alg:runtask}
\end{algorithm}}

\newcommand{\executetaskalgo}{%
\begin{algorithm}[H]
  \SetAlgoLined
\SetKwData{Left}{left}\SetKwData{This}{this}\SetKwData{Up}{up}
\SetKwFunction{Union}{Union}\SetKwFunction{FindCompress}{FindCompress}
\SetKwInOut{Input}{输入}\SetKwInOut{Output}{输出}
%\SetKwInOut{Input}{Input}\SetKwInOut{Output}{Output}
\SetKw{KwBreak}{Break}
\Input{需要执行的任务\code{task};}
\Input{任务\code{task}所依赖的前提条件,依赖表达式\code{DE};}
\Input{当前线程挂起任务队列\code{STQ}。}
%\Begin{
task.state = \state{running} \;
\eIf{DE.check\_if\_over()}{
  task.run()\;
  task.state = \state{over}\;
}
{
  STQ.push(task)\;
}
\end{algorithm}}

\newcommand{\suspendpopalgo}{%
\begin{algorithm}[H]
  \SetAlgoLined
\SetKwData{Left}{left}\SetKwData{This}{this}\SetKwData{Up}{up}
\SetKwFunction{Union}{Union}\SetKwFunction{FindCompress}{FindCompress}
\SetKwInOut{Input}{输入}\SetKwInOut{Output}{输出}
%\SetKwInOut{Input}{Input}\SetKwInOut{Output}{Output}
\SetKw{KwBreak}{Break}
\Input{任务, \code{t}。}
%\Begin{
h = head \;
\If {h == tail}{
  \KwRet invalid\_task\;
}
r = array[h\%length]\;
\While{!cas(\&head, h, h+1)}{
  h = head\;
  \If{h == tail}{
    \KwRet invalid\_task \;
  }
  r = array[h\%length]\;
}
\end{algorithm}}

\newcommand{\suspendpushalgo}{%
\begin{algorithm}[H]
  \SetAlgoLined
\SetKwData{Left}{left}\SetKwData{This}{this}\SetKwData{Up}{up}
\SetKwFunction{Union}{Union}\SetKwFunction{FindCompress}{FindCompress}
\SetKwInOut{Input}{输入}\SetKwInOut{Output}{输出}
%\SetKwInOut{Input}{Input}\SetKwInOut{Output}{Output}
\SetKw{KwBreak}{Break}
\Input{任务, \code{t}。}
%\Begin{
\If{head-tail $>$ length}{
  resize\_array()\;
}
queue[tail\%length] = t\;
tail ++\;
\end{algorithm}}

\newcommand{\schedulealgo}{%
\begin{algorithm}[H]
  \SetAlgoLined
\SetKwData{Left}{left}\SetKwData{This}{this}\SetKwData{Up}{up}
\SetKwFunction{Union}{Union}\SetKwFunction{FindCompress}{FindCompress}
\SetKwInOut{Input}{输入}\SetKwInOut{Output}{输出}
%\SetKwInOut{Input}{Input}\SetKwInOut{Output}{Output}
\SetKw{KwBreak}{Break}
\Input{当前线程所对应的任务队列, {$Q$};}
\Output{线程队列中的所有任务执行完毕。}
%\Begin{
  \While{is\_running}{
    $t$ $\leftarrow$ dequeue{($Q$)} \;
    t\_record\_lock($\cdots$)\; \label{code:alg:runtask:record}
    \eIf{t\_checker($\cdots$)}{ \label{code:alg:runtask:check}
      enqueue($Q$, $t$)\; \label{code:alg:runtask:enque}
    }{
      execute\_task($t$)\; \label{code:alg:runtask:execute}
    }\label{code:endif}
    t\_erase\_lock($\cdots$)\; \label{code:alg:runtask:erase}
  %}
}
\label{alg:runtask}
\end{algorithm}}

\newcommand{\recordalgo}{%
\begin{algorithm}[H]
%  \SetAlgoLined
\SetKwData{Left}{left}\SetKwData{This}{this}\SetKwData{Up}{up}
\SetKwFunction{Union}{Union}\SetKwFunction{FindCompress}{FindCompress}
\SetKwInOut{Input}{输入}\SetKwInOut{Output}{输出}
\SetKw{KwBreak}{break}
\SetKw{KwNull}{null}
\Input{锁$l$,以及全局锁集$LS$}
\Output{将锁$l$添加到全局锁集$LS$中。}
%\Begin{
  $i \leftarrow$ local\_threrad\_id\;
      \For{$j\leftarrow 0$ \KwTo len($LS[i]$)}{
        \If{$LS[i][j] = $ \KwNull}{
          $LS[i][j]$.atomic\_store($l$)\;
          \KwBreak
        }
      }
%}
\label{alg:recordlock}
\end{algorithm}}

\newcommand{\erasealgo}{%
\begin{algorithm}[H]
%  \SetAlgoLined
\SetKwData{Left}{left}\SetKwData{This}{this}\SetKwData{Up}{up}
\SetKwFunction{Union}{Union}\SetKwFunction{FindCompress}{FindCompress}
\SetKwInOut{Input}{输入}\SetKwInOut{Output}{输出}
\SetKw{KwBreak}{break}
\SetKw{KwNull}{null}
\Input{锁$l$,以及全局锁集$LS$}
\Output{将锁$l$从锁集$LS$中移除。}
%\Begin{
  $i \leftarrow$ local\_threrad\_id\;
      \For{$j\leftarrow 0$ \KwTo len($LS[i]$)}{
        \If{$LS[i][j] = l$}{
          $LS[i][j]$.atomic\_store(\KwNull)\;
          \KwBreak
        }
      }
%}
\label{alg:eraselock}
\end{algorithm}}

\newcommand{\islockalgo}{%
\begin{algorithm}[H]
%  \SetAlgoLined
\SetKwData{Left}{left}\SetKwData{This}{this}\SetKwData{Up}{up}
\SetKwFunction{Union}{Union}\SetKwFunction{FindCompress}{FindCompress}
\SetKwInOut{Input}{输入}\SetKwInOut{Output}{输出}
\Input{锁$l$,以及全局锁集$LS$}
\Output{如果$l \in LS$则返回{\true},否则返回 {\false}。}
%\Begin{
  $i \leftarrow$ local\_threrad\_id\;
  \For{$s\leftarrow 0$ \KwTo len($LS$)} {
    \If{$s\ne i$}{
      \For{$j\leftarrow 0$ \KwTo len($LS[s]$)}{
        \If{$LS[s][j] = l$}{
          \KwRet \true\;
        }
      }
    }
  }
  \KwRet \false\;
%}
\label{alg:islock}
\end{algorithm}}

\newcommand{\figwslb}{%
  \begin{minipage}{.4\textwidth}
      \begin{figure}[H]
        \scalebox{0.9}{
          \begin{tikzpicture}
            \begin{axis}[alllb]
\addplot [black, fill=lightgray, pattern=north east lines ] coordinates{(1, 71.2599619667)
(2, 67.1584026236)
(3, 76.5419934188)
(4, 77.7173795747)
(5, 83.8867704963)
(6, 84.8264400632)
};
\addplot [black, fill=black ] coordinates{(1, 68.3585750098)
(2, 66.807703796)
(3, 78.4629556951)
(4, 76.6173795747)
(5, 82.468485825)
(6, 84.7255123413)
};
\legend{{原始的FF}, {带SSLink的FF}}
\end{axis}

          \end{tikzpicture}}
        \caption{{\bf Load balancing of integrating {\name} with work
            stealing.}}
        \label{fig:lb}
      \end{figure}
      \end{minipage}
}

\newcommand{\figwscache}{%
  \begin{minipage}{.4\textwidth}
      \begin{figure}[H]
        \scalebox{0.9}{
         \begin{tikzpicture}
            \begin{axis}[cachemiss]
\addplot [black, fill=lightgray, pattern=north east lines ] coordinates{(1, 6.22958758302)
(2, 31.5467376457)
(3, 27.6197940831)
(4, 17.713108506)
(5, 23.310657737)
(6, 14.0689083247)
};
\addplot [black, fill=black ] coordinates{(1, 5.3362178602)
(2, 31.7773057901)
(3, 26.6210883928)
(4, 18.615012984)
(5, 23.4781826681)
(6, 13.7703185169)
};
\legend{{原来的FF}, {带{\name}的FF}}
\end{axis}

         \end{tikzpicture}}
        \caption{{\bf Cache miss rate of integrating {\name} with work
            stealing.}}
        \label{fig:cm}
      \end{figure}
      \end{minipage}
}

\newcommand{\figmicrovar}{%
  \begin{minipage}{.44\textwidth}
\begin{figure}[H]
  \centering
  \scalebox{0.9}{
  \begin{tikzpicture}
    \begin{axis}[microvarbar, ymode=log, log basis y={10}, log ticks with fixed point]
\addplot[black, fill=yellow, pattern=north east lines  , area legend] coordinates {(1, 2.4675844104e+12) (2, 4.6919565144e+12) (3, 4.7234738106e+12) (4, 4.8674077992e+12) };
\addplot[black, fill=black, area legend] coordinates {(1, 9.24503648828e+12) (2, 1.13255978625e+13) (3, 1.21591050883e+13) (4, 3.92176427094e+13) };
\legend{{原始FF}, {带{\name}的FF}}
\end{axis}

  \end{tikzpicture}}
  \caption{ {\bf Variance of lock acquisition for micro-benchmark.}}
  \label{fig:lockacq}
\end{figure}
\end{minipage}
}

\newcommand{\figholdmutex}{%
  \begin{minipage}{.44\textwidth}
\begin{figure}[H]
  \centering
  \scalebox{0.9}{
  \begin{tikzpicture}
    \begin{axis}[mhtime] 
\addplot [ blue, mark=o ] plot coordinates {
(11.9058713698, 1.32843098085)
(13.0302003636, 1.57242985254)
(14.3627849096, 1.31996021282)
(16.5968964485, 1.36328600156)
(20.7046009518, 1.35135994292)
(23.9654545126, 1.75032047781)
(31.5236583261, 1.89392075514)
(37.7539176922, 2.1427291967)
(48.5205133962, 2.57810814351)
(59.3428497569, 2.73537854903)
(69.9041874295, 2.98620323341)
(78.3027963381, 3.27717217643)
(85.3989921605, 3.66294578657)
(90.6102365531, 3.40097844173)
(93.4561059374, 3.66735257714)
(96.2896406664, 3.44887234862)
};\end{axis}

  \end{tikzpicture}}
\caption{{\bf Speedup v.s. lock holding time ratio.}}
\label{fig:mutexhold}
\end{figure}
\end{minipage}
}




\newcommand{\tablock}{%
  \begin{minipage}{.45\textwidth}
\begin{table}[H]
  \caption{{\bf Number of locks in each application.} O(N) for \bench{SSCA} means
    the number of locks depends on input size. In our evaluation of
    \bench{SSCA}, there are $2^{16}$ locks. \label{tab:locks}}%
\centering
 \begin{tabular}{|T{0.4\textwidth}|T{0.25\textwidth}|c|}
\hline
Category & Benchmark & \# of Locks \\
\hline
\multirow{3}{*}{Target Apps}  &
\bench{BQ} & 2 \\
& \bench{VN} & 4 \\
& \bench{SSCA} & O(N) \\
\hline
\multirow{3}{*}{Non-target Apps}  &
 \bench{CA} & 0 \\
& \bench{SC} & 1 \\
& \bench{QS} & 0 \\
\hline
\end{tabular}
\end{table}
\end{minipage}
}




