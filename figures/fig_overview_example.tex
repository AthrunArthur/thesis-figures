\begin{figure}
  \centering
\begin{lstlisting}[mathescape]
int fib(int n){
  if(n < MIN_PARA_N)  return serial_fib(n);
  ff::para<int> a, b;
  a([n](){return fib(n-1);});
  b([n](){return fib(n-2);});
  return (a&&b).then([](int m,int n){return m+n;});
}
\end{lstlisting}
\caption{A {\fname} program which recursively computes the \(n\)th Fibonacci
    number. Notice that this program is only used to show {\it how} to
    program with {\name}. And this is not a practical way to
    calculate Fibonacci numbers.}
\label{fig:overview}
\end{figure}
