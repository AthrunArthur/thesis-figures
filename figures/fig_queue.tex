\begin{figure}
\tikzset{
block/.style={draw, fill=blue!20, minimum width=9pt, minimum
height=13pt, node
distance=0pt}}

\begin{tikzpicture}
\node[block] (tail) {};
\node[block, right =of tail] (n1){};
\node[block, right=of n1] (n2) {};
\node[block, right=of n2] (n3) {};
\node[block, right=of n3] (n4) {};
\node[block, right=of n4] (n5) {};
\node[block, right=of n5] (n6) {};
\node[block, right=of n6] (head) {};

\node [below=1pt of head.south east, node distance=3pt] {\figfont{head}};
\node [below=1pt of tail.south west] {\figfont{tail}};
\node [above=1pt of n4.north] {\figfont{\bf work-stealing queue}};
\draw [-]  ($(tail.north west) + (-0.5, 0)$) -- (tail.north west);
\draw [-]  ($(tail.south west) + (-0.5, 0)$) -- (tail.south west);

\draw[->] (head.east) -- ($(head.east) + (30pt, 0)$) {} node[very near end, sloped, above] {\figfont{thief threads}};
\draw[->, dashed] ($(tail.north west) + (-1.75, 0)$)  .. controls ($(tail.north
west) + (-1.2, 0)$) .. ($(tail.north west) + (0, -4pt)$) {} node [very near start, sloped, above] {\figfont{local thread}};
\draw[<-] ($(tail.south west) + (-1.75, 0)$) .. controls ($(tail.south west) +
(-1.2, 0)$)  .. ($(tail.south west) + (0, 4pt)$) {} node [very near start,
sloped, below] {\figfont{local thread}};

\node[block, right=150pt of head] (atail) {};
\node[block, right =of atail] (an1){};
\node[block, right=of an1] (an2) {};
\node[block, right=of an2] (an3) {};
\node[block, right=of an3] (an4) {};
\node[block, right=of an4] (an5) {};
\node[block, right=of an5] (an6) {};
\node[block, right=of an6] (ahead) {};

\node [below=1pt of ahead.south east, node distance=3pt] {\figfont{head}};
\node [below=1pt of atail.south west] {\figfont{tail}};
\node [above=1pt of an4.north] {\figfont{\bf suspended-task queue}};
\draw [-]  ($(atail.north west) + (-0.5, 0)$) -- (atail.north west);
\draw [-]  ($(atail.south west) + (-0.5, 0)$) -- (atail.south west);

\draw[->] ($(ahead.east) + (0, 1.5pt)$) -- ($(ahead.east) + (30pt, 1.5pt)$) {} node[very near end, sloped, above] {\figfont{thief threads}};
\draw[->] ($(ahead.east) + (0, -1.5pt)$) -- ($(ahead.east) + (30pt, -1.5pt)$) {}
node[very near end, sloped, below] {\figfont{local thread}};
\draw[->, dashed] ($(atail.west) + (-1.75, 0)$) -- (atail.west) {}
node [very near start, sloped, above] {\figfont{local thread}};

\coordinate  (A) at ($(tail.south west) + (1.5, -20pt)$);
\coordinate [label=right:{{\figfont{put a task into the task queue}}}] (B) at
($(tail.south west) + (2.0, -20pt)$);
\draw [->, dashed] ( A) --(B);

\coordinate  (C) at ($(atail.south west) + (-1.5, -20pt)$);
\coordinate [label=right:{{\figfont{pop a task from the task queue}}}]
(D) at ($(atail.south west) + (-1, -20pt)$){};
\draw[->] (C) --(D);


\end{tikzpicture}
\caption{The work-stealing queue and suspended-task queue in {\name}.
  Notice that each thread has its own work-stealing queue and
  suspended-task queue. Also note that work-stealing queue and
  suspended-task queue have different behaviors on pushing and poping,
  which give them different performance.}
\label{fig:queue}
\end{figure}
