\setlength{\tabcolsep}{1pt}
\normalsize
\begin{tabularx}{1.25\textwidth}{|c|C|C|C|C|C|C|C|C|}
\hline \hline
\multirow{2}{*}{任务协调模式}& \multicolumn{4}{|c|}{
  并行编程库} & \multicolumn{4}{|c|}{并行编程语言} \\
\cline{2-9}
& TBB& TPL & PPL & pFunc &
Cilk & OpenMP & X10 & Chapel \\
\hline
等待子任务& \cmark   & \cmark    & \cmark &  \cmark & \cmark  & \cmark &
\cmark & \cmark \\
等待所有任务 & \xmark   &  \cmark  &   \cmark &  \cmark    & \cmark   & \xmark &
\xmark &\xmark \\
等待任一任务 & \xmark   &  \cmark   &  \cmark  &  \cmark    & \xmark   & \xmark
&\xmark &  \xmark \\
组合等待& \tmark & \xmark   & \xmark & \xmark     &  \xmark  & \xmark &\xmark
&\xmark \\
\hline
\multicolumn{9}{l}{%
  \begin{minipage}{0.95\textwidth}
  \footnotesize {\cmark}: 支持指定的协调模式; \qquad
  \footnotesize {\xmark}: 不支持\\
  \footnotesize {\tmark}:
  TBB使用额外的组件进行支持,包括\code{tbb::pipeline}及\code{tbb::flowgraph};
  \end{minipage}
}
\end{tabularx}
